\documentclass{article}

\title{Homework 1}
\author{Chandler Swift}
\date{September 11, 2019}
\usepackage{hyperref}
\usepackage{amsmath}
\begin{document}
\maketitle
\begin{enumerate}
  \item Text problem 1.8. Simply plugging in 2025 yields $8.00 \times
    10^{-0.05806(2025-1970)} \approx 5.13\text{nm} = 51.3\text{\AA}$. This is
    about the width of 9-10 Silicon atoms, and indeed only slightly smaller
    than the process nodes currently being used at about 7nm (Apple, AMD) or
    Intel's 10nm silicon.

    However, given that we're somewhat ahead of the curve given (in 2019,
    seeing 7nm technology instead of the 11nm predicted by the equation for
    2019, we notice that the curve is slightly underestimating the rate of
    increase in transistor density. We will not likely be able to continue to
    see similar increases in transistor density in the future, as the width
    of transistors cannot get much smaller than they are currently.

  \item A-D Conversion Table:
    % for i in range(16):
    %   print(f"      {i:04b} & {i/16*5}-{(i+1)/16*5}V \\\\")
    \begin{tabular}{ c | c }
      Digital Output & Analog Input (Volts) \\ \hline
      0000 & 0.0-0.3125V \\
      0001 & 0.3125-0.625V \\
      0010 & 0.625-0.9375V \\
      0011 & 0.9375-1.25V \\
      0100 & 1.25-1.5625V \\
      0101 & 1.5625-1.875V \\
      0110 & 1.875-2.1875V \\
      0111 & 2.1875-2.5V \\
      1000 & 2.5-2.8125V \\
      1001 & 2.8125-3.125V \\
      1010 & 3.125-3.4375V \\
      1011 & 3.4375-3.75V \\
      1100 & 3.75-4.0625V \\
      1101 & 4.0625-4.375V \\
      1110 & 4.375-4.6875V \\
      1111 & 4.6875-5.0V \\
    \end{tabular}
  \item $2^{24} = 16777216$. Experimentally, I can't see this number of colors.
    (To test I built a demo page at \url{https://duluth.chandlerswift.com/color.html}
    which gives comparisons of color changed by one bit. They're not visually
    distinguishable, at least to my eyes on my screen.

    $2^{30} = 1073741824$, which should certainly be indistinguishable, if 
    24bpp (bit-per-pixel) color is not.
  \item
    \begin{enumerate}
      \item 
        \begin{enumerate}
          \item V\textsubscript{th}=3.33V
          \item R\textsubscript{th}=3.33k$\Omega$
        \end{enumerate}
      \item
        \begin{enumerate}
          \item V\textsubscript{th}=9V
          \item R\textsubscript{th}=5k$\Omega$
        \end{enumerate}
    \end{enumerate}
  \item
    \begin{enumerate}
      \item
        V\textsubscript{th}=89.6V$_i$ \\
        R\textsubscript{th}=75
      \item
        V\textsubscript{th}= \\
        R\textsubscript{th}=75
    \end{enumerate} 
  \item 
    \begin{tabular}{ l | l }
      Parameter & equation \\ \hline
        $V_{ba}(t)$ & $\sin(170 \pi t)V$ \\ \hline
        $V_{ca}(t)$ & $0V$ \\ \hline
        $V_{bc}(t)$ & $\sin(170 \pi t)V$ \\ \hline
        $V_{ba}$(Peak Phasor) & $170V$ \\ \hline
        $V_{ca}$(Peak Phasor) & $0V$ \\ \hline
        $V_{bc}$(Peak Phasor) & $170V$ \\ \hline
        $V_{ba}$(RMS Phasor) & $120V$ \\ \hline
        $V_{ca}$(RMS Phasor) & $0V$ \\ \hline
        $V_{bc}$(RMS Phasor) & $120V$ \\ \hline
    \end{tabular}
  \item
    \begin{enumerate}
      \item
        \begin{enumerate}
          \item The black wire (hot) is connected to the left prong, and white
            (neutral) to the right prong. These should be switched. 
          \item The fuse is connected on the grounded side, and should be
            connected on the upstream side so that when it is blown there is no
            power at the outlet (since if the ground was connected, that would
            be a complete circuit, even without the neutral connector in place).
          \item The ground is not connected. 
          \item The ground is red instead of green.
        \end{enumerate}
      \item
        \begin{enumerate}
          \item The ground not being connected could make you the easiest path
            to ground!
          \item With the fuse on the neutral side, the fuse could blow and
            there could still be power at the outlet, including a complete
            circuit if some path to ground was present.
          \item With hot and neutral switched, appliances expecting neutral to
            be a safe voltage level might not take the same level of care
            preventing it from coming in contact with the user (e.g. switching
            the incorrect side).
        \end{enumerate}
      \item The lamp will work---the correct voltage is still being provided
        to its two contacts, albeit backwards. However, being an AC circuit,
        there's no such thing as backwards!
    \end{enumerate}
  \item
    \begin{enumerate}
      \item 
        \begin{enumerate}
          \item 41.6dB
          \item 35.6dB
          \item 94.0dB
          \item 100dB
          \item -0.915dB
        \end{enumerate}
      \item ---
      \item
        \begin{enumerate}
          \item 93dB
          \item 56dB
          \item 88dB
          \item 110dB
        \end{enumerate}
    \end{enumerate}
  \item 
    \begin{enumerate}
      \item $\frac{(10\times10^{-6})^2}{50} = 2\times10^{-12}$
      \item $10 \log (\frac{200}{2 \times 10^{-12}}) \approx 322$dB
      \item 40V
      \item $20 \log(\frac{40V}{10 \times 10^{-6}V}) \approx 304$dB
    \end{enumerate}
\end{enumerate}

\end{document}

